\documentclass[a4paper,12pt]{article}
\usepackage[slovene]{babel} % jeziki
\usepackage[utf8]{inputenc} % lahko pišem čšž
\usepackage[T1]{fontenc} % prikaze čšž
\usepackage{lmodern}
\usepackage{amsmath}
\usepackage{eurosym}
\usepackage{units}
\usepackage{mathtools}
\usepackage{amssymb}
\usepackage{amsthm}
\usepackage{enumerate}
\usepackage{graphicx}
\usepackage{multirow}
\usepackage{array}
\usepackage{amsfonts}
\usepackage{url}
\usepackage{color}
\usepackage[nottoc]{tocbibind}
\usepackage[pdftitle={Moj naslov},bookmarks=false,urlbordercolor={1 1 1},linkbordercolor={1 1 1},hidelinks]{hyperref}

\newcommand{\N}{\mathbb{N}}
\newcommand{\R}{\mathbb{R}}
\newcommand{\Q}{\mathbb{Q}}
\newcommand{\Z}{\mathbb{Z}}
\newcommand{\va}{\vec{a}}
\newcommand{\vb}{\vec{b}}
\newcommand{\vc}{\vec{c}}
\newcommand{\vd}{\vec{d}}
\newcommand{\vr}{\vec{r}}
\newcommand{\pojem}[1]{\textsc{#1}}
\newcommand{\equaltext}[1]{\ensuremath{\stackrel{\text{#1}}{=}}}

\theoremstyle{definition}
\newtheorem{df}{Definicija}[section]

\theoremstyle{proposition}
\newtheorem{trd}{Trditev}[section]

\theoremstyle{theorem}
\newtheorem{iz}{Izrek}[section]

\theoremstyle{lemma}
\newtheorem{lem}{Lema}[section]

\newtheorem*{op}{Opomba}

\title{\textbf{\huge{Verižni ulomki}\\\normalsize{Seminarska naloga}}}
\author{Gašper Urh}
\date{\today}

\begin{document}

\begin{titlepage}
	\centering
	{\scshape\LARGE Univerza v Ljubljani, Fakulteta za matematiko in fiziko\par}
	\vspace{1cm}
	{\scshape\Large Seminarska naloga\par}
	\vspace{1.5cm}
	{\huge\bfseries Verižni ulomki\par}
	\vspace{2cm}
	{\Large\itshape Gašper Urh\par}

	\vfill

	{\textit{Mentor: dr. David Dolžan}}\par
	\vspace{1cm}
	{\large \today\par}
\end{titlepage}

\tableofcontents
\newpage

\section{Uvod}

Verižni ulomki so za decimalnim zapisom manj uporabljana predstavitev realnih števil, a so kljub temu zelo koristni tako za reševanje nekaterih problemov teorije števil kot tudi za iskanje racionalnih približkov realnih (predvsem iracionalnih) števil. Z njimi se da na matematično lep način karakterizirati kvadratne iracionale, iskati rešitve diofantske enačbe in dokazati nekatere zelo pomembne izreke iz teorije števil. \par
V svoji seminarski nalogi sem se poglobil predvsem v teoretično ozadje zapisa števila z verižnim ulomkom.\par
V poglavju \ref{ospojmi} bomo definirali verižni ulomek in nekatere osnovne pojme, ki jih bomo uporabljali v nadaljevanju, ter se seznanili z izgledom verižnega ulomka. Poglavje \ref{konv} je posvečeno pojmu konvergenta, nekaterim formulam in pojmom, ki jih bomo v nadaljevanju uporabljali, ter konvergenci enostavnega verižnega ulomka. S poglavjem \ref{racionalna} si bomo ob racionalnih številih ogledali postopek iskanja verižnega ulomka, kar bomo v poglavju \ref{realna} razširili na iskanje verižnih ulomkov za poljubna realna števila. V poglavju \ref{euler} bomo izpeljali verižni ulomek za Eulerjevo število $e$ in v poglavju \ref{periode} pokazali povezavo med periodičnimi verižnimi ulomki in kvadratnimi iracionalami.

\section{Osnovni pojmi} \label{ospojmi}

\begin{df}
	\textit{Verižni ulomek} je izraz oblike
	\[ a_0 + \cfrac{1}{a_1 + \cfrac{1}{a_2 + \cfrac{1}{a_3 + \cfrac{1}{...}}}}, \]
	pri čemer privzemimo, da so $a_n$ realna števila, ki so za $n \geq 1$ tudi pozitivna. Verižni ulomek je lahko bodisi končen, bodisi neskončen.
	\newline
	\newline
	Zapišemo ga lahko tudi kot $[a_0, a_1, a_2, a_3, ...]$.
\end{df}
	
Iz definicije in oblike zapisa verižnega ulomka neposredno sledi naslednja ugotovitev:

\begin{equation}
\label{ocitno}
[a_0, a_1, ..., a_n, [a_{n+1}, ...]]=\Bigg[ a_0, a_1, ..., a_n+\frac{1}{[a_{n+1}, ...]}\Bigg]
\end{equation}

V literaturi pogosto obravnavajo samo posebno lepe verižne ulomke. Take bomo v tem besedilu imenovali \textit{enostavni verižni ulomki}. Večinoma se bomo ukvarjali le s takimi.

\begin{df}
	$[a_0, a_1, a_2, ...]$ je \textit{enostaven}, če $a_0 \in \Z$ ter $a_1, a_2, ... \in \N$
\end{df}

Da dobimo občutek, kako je videti racionalno število, zapisano z verižnim ulomkom, izračunajmo \[ [3,6,4] = 3 + \cfrac{1}{6+\cfrac{1}{4}} = \frac{79}{25}.\]
Vidimo, da je za racionalna števila razmeroma enostavno najti vrednost verižnega ulomka. Kasneje bomo pokazali, da je tak verižni ulomek vedno končen.\\ Veliko zanimivejše je iskanje verižnega ulomka iracionalnega števila. Število $\pi$ lahko zapišemo kot neskončen verižni ulomek \[ \pi = 3 + \cfrac{1}{7 + \cfrac{1}{15 + \cfrac{1}{1 + \cfrac{1}{292 + \cfrac{1}{...}}}}}. \]
Kako priti do takega zapisa, bomo videli v poglavju \ref{realna}.

\section{Konvergenti} \label{konv}

V tem poglavju bomo opazovali, ali lahko nek neskončen verižni ulomek, za katerega velja $a_n \in \Z$ za vsak $n = 0, 1, 2, 3, ...$, predstavlja neko število. V ta namen si bomo ogledali zaporedje njegovih \textit{delnih konvergentov}.

\begin{df}
Naj bo $0 \leq n \leq m$. Število $c_n$ je $n$-ti \textit{konvergent} 		verižnega ulomka $[a_0, a_1, ..., a_m]$, če je $c_n = [a_0, a_1, ..., a_n]$. Za $n < m$ je to \textit{delni konvergent}.
\end{df}

Izračunajmo nekaj delnih konvergentov verižnega ulomka $[a_0, a_1, a_2, ...]$: 
\begin{equation*}
c_0=a_0,
\end{equation*}
\begin{equation}
\label{prvikonvergenti}
c_1=a_0+\cfrac{1}{a_1}=\frac{a_1 a_0 + 1}{a_1}
\end{equation}
\begin{equation*}
c_2 = a_0 + \cfrac{1}{a_1 + \cfrac{1}{a_2}} =\frac{a_0 a_1 a_2 + a_0 + a_2}{a_1 a_2 + 1}
\end{equation*}

Bolj ali manj očitno gre pri števcih in imenovalcih konvergentov za neko rekurzijo. Definirajmo dve zaporedji za $-2\leq n \leq m$:
\begin{align*}
		p_{-2} &= 0, & p_{-1} &= 1, & p_0 &= a_0, & \dots, \quad p_n &= a_{n}p_{n-1} + p_{n-2}, & \dots \\
		q_{-2} &= 1, & q_{-1} &= 0, & q_0 &= 1, & \dots, \quad q_n &= a_{n}q_{n-1} + q_{n-2}, & \dots
\end{align*}
		
\begin{trd}
Naj bo $0\leq n \leq m$. Tedaj je
\begin{equation}
\label{pninqn}
c_n = [a_0, ..., a_n] = \frac{p_n}{q_n} = \frac{a_{n}p_{n-1} + p_{n-2}}{a_{n}q_{n-1} + q_{n-2}}.
\end{equation}
\end{trd}

\begin{op}
Vemo, da je $a_1 \geq 1$, torej je $q_1 \geq 1$. Induktivno dobimo
\begin{equation}
\label{ocenaimenovalca}
q_n \geq n
\end{equation}
\end{op}

\begin{proof}
Dokazujemo z indukcijo. Za $a_0, a_1$ vidimo resničnost trditve s pomočjo zgoraj izračunanih konvergentov (\ref{prvikonvergenti}). Privzemimo, da trditev velja za verižne ulomke dolžine $n$. Tedaj je
\begin{equation*}
\begin{split}
c_{n+1}&=[a_0, ...,a_n, a_{n+1}]
\equaltext{\text{*}}{} \Big[a_0, ..., a_n + \frac{1}{a_{n+1}}\Big] =\\
&= \frac{\big(a_n + \frac{1}{a_{n+1}}\big)p_{n-1}+p_{n-2}}{\big(a_n + \frac{1}{a_{n+1}}\big)q_{n-1}+q_{n-2}} =
\frac{a_{n+1}(a_n p_{n-1} + p_{n-2}) + p_{n-1}}{a_{n+1}(a_n q_{n-1} + q_{n-2}) + q_{n-1}} =\\
&=\frac{a_{n+1}p_n + p_{n-1}}{a_{n+1}q_n + q_{n-1}} = \frac{p_{n+1}}{q_{n+1}}.
\end{split}
\end{equation*}
Enakost * sledi iz (\ref{ocitno}).
\end{proof}

\vspace{5mm}

Iz zgornje rekurzije sledita še dve formuli:
\begin{equation}
\label{prvaformula}
p_n q_{n-1} - q_n p_{n-1} = (-1)^{n-1}
\end{equation}
in
\begin{equation}
\label{drugaformula}
p_n q_{n-2} - q_n p_{n-2} = (-1)^n a_n,
\end{equation}

ki sta ekvivalentni formulama

\begin{equation}
\label{prvaformulaekv}
\frac{p_n}{q_n} - \frac{p_{n-1}}{q_{n-1}} = (-1)^{n-1}\frac{1}{q_n q_{n-1}}
\end{equation}
ter
\begin{equation}
\label{drugaformulaekv}
\frac{p_n}{q_n} - \frac{p_{n-2}}{q_{n-2}} = (-1)^{n}\frac{a_n}{q_n q_{n-2}}.
\end{equation}

\begin{proof}
Dokazali bomo samo formulo (\ref{prvaformula}). Na podoben način dokažemo (\ref{drugaformula}), formuli (\ref{prvaformulaekv}) in (\ref{drugaformulaekv}) pa dobimo iz (\ref{prvaformula}) in (\ref{drugaformula}) s primernim deljenjem enačbe.
Bazo indukcije za (\ref{prvaformula}) dokažemo brez težav, le z vstavljanjem ustreznih vrednosti. Privzemimo sedaj, da fomula velja za $n-1$, torej 
\[ p_{n-1} q_{n-2} - q_{n-1} p_{n-2} = (-1)^{n-2}. \]
\begin{equation*}
\begin{split}
p_n q_{n-1} - q_n p_{n-1} &= (a_{n}p_{n-1} + p_{n-2}) q_{n-1} - (a_{n}q_{n-1} + q_{n-2}) p_{n-1} = \\
&= a_{n}p_{n-1}q_{n-1} + p_{n-2} q_{n-1} - a_{n}q_{n-1}p_{n-1} - q_{n-2} p_{n-1} = \\
&= p_{n-2} q_{n-1} - q_{n-2} p_{n-1} = \\
&= -(-1)^{n-2} = (-1)^{n-1}.
\end{split}
\end{equation*}
\end{proof}

Oglejmo si ulomek $\frac{p_n}{q_n}$. Naj bo $d$ največji skupni delitelj števil $p_n$ in $q_n$. Tedaj $d$ deli $p_n q_{n-1} - q_n p_{n-1}$ in po (\ref{prvaformula}) deli $(-1)^{n-1}$. Torej $d= \pm 1$ in števili sta si tuji. Torej je $c_n$  pridobljen s (\ref{pninqn}) vedno okrajšan ulomek.\\

\subsection{Konvergenca enostavnega verižnega ulomka} \label{konvca}

Izračunajmo nekaj konvergentov verižnega ulomka za $\pi$.

\[ \pi = [3, 7, 15, 1, 292, 1, ...] \]
Konvergenti so:
	\begin{itemize}
	\item $c_0 = 3$
	\item $c_1 = \frac{22}{7} = 3,1428571$
	\item $c_2 = \frac{333}{106} = 3,141509434$
	\item $c_3 = \frac{355}{113} = 3,14159292$
	\item $c_4 = \frac{103993}{33102} = 3,141592653$
	\item $c_5 = \frac{104348}{33215} = 3,141592654$
	\item ...
	\end{itemize}
	
Vidimo, da zaporedje prvih nekaj sodih konvergentov narašča, lihih pa pada.

\begin{trd}
\label{podzaporedji}
Zaporedje sodih konvergentov je strogo naraščajoče, zaporedje lihih pa strogo padajoče. Za vsak $n, m \in \N$ je  $c_{2n} < c_{2m+1}$.
\end{trd}
\begin{proof}
Po formuli (\ref{drugaformulaekv}) je razvidno, da je zaporedje $\{c_{2n}\}$ naraščajoče, $\{c_{2n-1}\}$ pa padajoče. \\
Privzemimo, da obstajata $n, m$, da je $c_{2n} > c_{2m+1}$. Iz (\ref{prvaformulaekv}) sledi $c_{2r} < c_{2r+1} \forall r \in \N$. Torej $ n \neq m$. Denimo, da je $n < m$. Tedaj je $c_{2m+1} < c_{2n} < c_{2m}$, kar je protislovje, saj dobimo $c_{2m+1} < c_{2m}$. Podobno, če $n > m$, je $c_{2n+1} < c_{2m+1} < c_{2n}$, kar zopet vodi v protislovje.
\end{proof}

\begin{iz}
\label{konvergenca}
Naj bo $[a_0, a_1, ...]$ enostaven verižni ulomek in naj bo za vsak $n$ $c_n = [a_0, a_1, ..., a_n]$ konvergent. Tedaj obstaja $$\lim_{n\to\infty}c_n.$$
\end{iz}

\begin{proof}
Videli smo, da je zaporedje sodih konvergentov strogo naraščajoče in zaporedje lihih strogo padajoče. Za obe zaporedji velja, da sta omejeni, saj je vsak sodi konvergent spodnja meja za zaporedje lihih in obratno. Torej obstajata $\lim_{n\to\infty}c_{2n}$ in $\lim_{n\to\infty}c_{2n-1}$. Pokažimo, da sta enaki. To sledi iz (\ref{prvaformulaekv}) in (\ref{ocenaimenovalca}):
\[ \left|c_{2n} - c_{2n-1}\right| = \frac{1}{q_{2n} q_{2n-1}} \leq \frac{1}{2n(2n-1)} \longrightarrow 0 \]
\end{proof}

Pokazali smo, da za vsak enostaven verižni ulomek obstaja limita zaporedja njegovih delnih konvergentov. Vsak enostaven verižni ulomek torej natanko določa neko število.

\section{Evklidov algoritem in verižni ulomki racionalnih števil} \label{racionalna}

Racionalna števila so razmeroma enostavna za zapisovanje z verižnim ulomkom, saj je njihov zapis vedno končen, kar nam zagotovi naslednja trditev.

\begin{trd}
\label{lastnosracionalnega}
Neko število je racionalno natanko tedaj, ko njegov zapis z verižnim ulomkom obstaja in je končen.
\end{trd}

\begin{proof}
($\Leftarrow$) Očitno je vsak končen verižni ulomek z nekaj dela mogoče zapisati kot enostaven ulomek, kar pomeni, da predstavlja racionalno število. Uporabimo (\ref{pninqn}).\\
($\Rightarrow$) Uporabimo Evklidov algoritem. Naj bo $\frac{a}{b}$ nek okrajšan ulomek (torej je največji skupni delitelj $a$ in $b$ enak $1$).\\
Zapišimo
\begin{align*}
a &= a_0 b + r_1 \\
b &= a_1 r_1 + r_2 \\
r_1 &= a_2 r_2 + r_3 \\
&\vdots \\
r_{n-2} &= a_{n-1} r_{n-1} + 1 \\
r_{n-1} &= a_n + 0.
\end{align*}

\vspace{5mm}

S primernim deljenjem enačb dobimo
\begin{align*}
a/b &= a_0 + \frac{1}{b/r_1} \\
b/r_1 &= a_1 + \frac{1}{r_1/r_2} \\
r_1/r_2 &= a_2 + \frac{1}{r_2/r_3} \\
&\vdots \\
r_{n-2}/r_{n-1} &= a_{n-1} + \frac{1}{r_{n-1}}, \\
\end{align*}
pri čemer operiramo zgolj s celimi števili. Dobimo ravno zapis $\frac{a}{b}$ s končnim enostavnim verižnim ulomkom
\end{proof}

\subsection{Enoličnost končnega enostavnega verižnega ulomka} \label{enolicnost}

\begin{lem}
\label{celidel}
Naj bo $[a_0, a_1, ..., a_m]$ enostaven verižni ulomek in naj bo $a_m \neq 1$, $m \geq 1$. Tedaj velja $0<x-a_0<1$.
\end{lem}
\begin{proof}
Velja $\frac{1}{x-a_0} = [a_1, ..., a_m]$. Ker velja $a_n \geq 1$ za pozitivne $n$, je $x-a_0 > 0$. Če je $m=1$, je $a_1 \geq 2$ in $\frac{1}{x-a_0} \leq \frac{1}{2}$, torej trditev velja. Če je $m\geq2$, pa je $ [a_2, ..., a_m] > 0$ in velja $\frac{1}{x-a_0} = [a_1, ..., a_m] > a_1 \geq1$, zato je $x-a_0<1$.
\end{proof}

\begin{trd}
\label{enolicnost}
Če lahko $x$ zapišemo s končnim enostavnim verižnim ulomkom, je ta s pogojem $a_m \neq 1$ enolično določen.
\end{trd}
\begin{proof}
Naj bo $x=[a_0, a_1, ..., a_m]=[b_0, b_1, ..., b_p]$. Brez škode za splošnost je $m\leq p$. Za $m>0$ velja $0<x-a_0<1$ in $0<x-b_0<1$ po Lemi \ref{celidel}, torej sta $a_0$ in $b_0$ enaka celemu delu $x$. Tedaj s primernim odštevanjem in invertiranjem zgornje enačbe dobimo $[a_1, a_2, ..., a_m]=[b_1, b_2, ..., b_p]$. Postopek ponavljamo in dobimo $a_m =[b_m, b_{m+1}, ..., b_p]$. Če $m \neq p$, pridemo v protislovje z dejstvom, da je $a_m \in \Z$. \\
Če je $m=0$, je $x$ celo število in velja $x-b_0 \in \Z$. Če $p\geq1$, je $0<x-b_0<1$, kar vodi v protislovje. Torej $x=a_0=b_0$. 
\end{proof}

\begin{op}
Kaj se zgodi, če ne zahtevamo, da je $a_m \neq 1$? Tedaj lahko enostaven neničelen verižni ulomek $[a_0, ..., a_{m-1}, 1]$ zapišemo tudi kot $[a_0, ..., a_{m-1}+1]$ in dobimo enostaven verižni ulomek, kjer je $a_{m-1}\neq1$, tak pa je za neko racionalno število natanko določen. Torej pri sprostitvi zahteve $a_m\neq 1$ dobimo natanko dva zapisa neničelnega racionalnega števila z verižnim ulomkom. $$\frac{5}{2}=[2,2]=[2,1,1]$$
\end{op}

\section{Postopek za iskanje verižnega ulomka realnega števila} \label{realna}

Oglejmo si primer, ko moramo z verižnim ulomkom zapisati poljubno realno število. V tem primeru Evklidov algoritem seveda ne deluje. Opisati želimo najbolj intuitiven postopek za zapis takega števila.\\
\vspace{5mm}
Naj bo $x \in \R$.
\[ x = a_0+t_0, \]
kjer je $a_0 \in \Z$ ter $t_0 \in [0,1)$.\\
Če $t_0 \neq 0$, zapišemo \[ \frac{1}{t_0} = a_1 + t_1. \]
Postopek z $a_n \in \N$ za $n \geq 1$ ponavljamo, dokler $t_n \neq 0$ (zapis je lahko neskončen).
\[ \frac{1}{t_n} = a_{n+1}+t_{n+1} \] \vspace{5mm}

\textbf{Zgled.} Vsem gotovo zelo znan je verižni ulomek, ki predstavlja razmerje zlatega reza, torej \[x = \frac{1+\sqrt{5}}{2}. \] Po zgornjem postopku zapišemo to kot celi del in ostanek. \[x = 1 +\frac{-1+\sqrt{5}}{2} \] Ostanek $t_0 \neq 0$ invertiramo in racionaliziramo imenovalec. Dobimo \[\frac{1}{t_0} =\frac{1+\sqrt{5}}{2} = x. \] Torej velja \[x = 1 + t_0 = 1 + \frac{1}{x} = 1 + \cfrac{1}{1+\cfrac{1}{1+\cfrac{1}{1+\ldots}}}.\]

Za $x$ iz zgornjega postopka sledi tudi naslednja lema.

\begin{lem}
\label{lema}
Za vsak $n$, za katerega je $t_n \neq 0$, velja
\[ x=[a_0, a_1, ..., a_n, \frac{1}{t_n}]. \]
\end{lem}
\begin{proof}
Dokazujemo z indukcijo. Za $n=0,1$ trditev očitno velja. Naj trditev velja za $n-1$. Tedaj je
\begin{equation}
\begin{split}
x &= \big[a_0,...,a_{n-1},\frac{1}{t_{n-1}}\big] \\
&= [a_0,...,a_{n-1}, a_n + t_n] \\
&= \big[a_0, ...,a_{n-1}, a_n, \frac{1}{t_n}\big].
\end{split}
\end{equation}
\end{proof}

Pokazali smo torej postopek, s katerim s pomočjo $x$ zapišemo nek verižni ulomek. Intuitivno se nam zdi, da ta ulomek tudi resnično konvergira k $x$.

\begin{iz}
\label{postopek}
Naj bo $x \in \R$. Tedaj je $x$ vrednost verižnega ulomka, ki ga dobimo z opisanim postopkom.
\end{iz}
\begin{proof}
Uporabimo Lemo \ref{lema} in zapišimo
\[ x =\big[a_0,..., a_n, \frac{1}{t_n}\big].\]
Sedaj uporabimo (\ref{pninqn}), kjer upoštevamo njeno veljavnost ne glede na pogoj, da so vsi $a_n$ cela števila.
\[x=\frac{\big(\frac{1}{t_n}\big)p_n+p_{n-1}}{\big(\frac{1}{t_n}\big)q_n+q_{n-1}}\]
Sedaj izračunajmo razliko delnega konvergenta $c_n$ in vrednosti $x$. Dobimo
\begin{equation*}
\begin{split}
\left|x-c_n\right|   &=\left|\frac{\big(\frac{1}{t_n}\big)p_n+p_{n-1}}{\big(\frac{1}{t_n}\big)q_n+q_{n-1}}-\frac{p_n}{q_n}\right|\\
&= \left|\frac{\frac{1}{t_n}p_n q_n + p_{n-1}q_n - \frac{1}{t_n}q_n p_n - q_{n-1} p_n}{\big(\frac{1}{t_n}q_n+q_{n-1}\big)q_n}\right| \\
&= \left|\frac{p_{n-1}q_n - q_{n-1} p_n}{\big(\frac{1}{t_n}q_n+q_{n-1}\big)q_n}\right| \\
&\equaltext{\text{*}}{} \frac{1}{\big(\frac{1}{t_n}q_n+q_{n-1}\big)q_n} \\
&\leq \frac{1}{\big(a_{n+1}q_n + q_{n-1}\big)q_n} \\
&= \frac{1}{q_{n+1}q_n} \\
&\leq \frac{1}{(n+1)n} \longrightarrow 0.
\end{split}
\end{equation*}
V (*) smo uporabili formulo (\ref{prvaformula}). V prvem neenačaju smo uporabili, da je $\frac{1}{t_n} = a_{n+1}+t_{n+1}$, torej je $a_{n+1} \leq \frac{1}{t_n}$.
\end{proof}

Z zgoraj opisanim postopkom torej dobimo enostaven (morda neskončen) verižni ulomek. V poglavju \ref{konvca} smo dokazali, da tak ulomek konvergira, zdaj pa smo dokazali, da vedno konvergira k številu, ki smo si ga izbrali za začetek izvajanja našega postopka za iskanje verižnega ulomka realnega števila.\par
Neenostavni ulomki lahko tudi divergirajo. Za ugotovitev, ali taki ulomki konvergirajo, uporabimo naslednjo trditev.

\begin{trd}
\label{divergenca}
Naj bo $a_0, a_1, a_2, ...$ zaporedje realnih števil, pri čemer je $a_n > 0$ za $n\geq1$. Naj bo $c_n = [a_0, a_1, a_2, ..., a_n]$. Tedaj obstaja $\lim_{n\to\infty}c_n$ natanko tedaj, ko $\sum_{n=0}^{\infty}a_n$ divergira.
\end{trd}
\begin{proof}
Dokaz najdemo v \cite{teorijastevil}, str. 105.
\end{proof}

Ta trditev je posplošitev izreka o konvergenci enostavnega verižnega ulomka, saj vrsta $\sum_{n=0}^{\infty}a_n$, kjer so $a_n$ naravna števila za $n \geq 1$, vedno divergira (členi te vrste se namreč ne približujejo $0$).

\section{Verižni ulomek števila $e$} \label{euler}

V tem poglavju si bomo ogledali posebno zaporedje $a_0,a_1,...$, ki tvori verižni ulomek števila $e$. Izračunajmo prvih nekaj členov tega zaporedja:
\[e = [2, 1, 2, 1, 1, 4, 1, 1, 6, 1, 1, 8, ...] = [1, 0, 1, 1, 2, 1, 1, 4, 1, 1, 6, 1, 1, 8, ...] \]
Zaradi lažje razpoznavnega zaporedja bomo tokrat uporabili $a_1=0$ in svojo naslednjo domnevo naslonili na drugo zaporedje. Dobimo tri podzaporedja:
\begin{align*}
a_{3n} &= 1 &\Rightarrow \quad & p_{3n} = p_{3n-1}+p_{3n-2} \hfill \\
a_{3n-1} &= 1 &\Rightarrow \quad & p_{3n-1} = p_{3n-2}+p_{3n-3} \hfill \\
a_{3n-2} &= 2(n-1) &\Rightarrow \quad & p_{3n-2} = 2(n-1)p_{3n-3}+p_{3n-4} \hfill \\
\end{align*}

Enaka rekurzija velja tudi za zaporedje $\{q_n\}$.\par
Za tako zaporedje $\{a_n\}$ želimo dokazati, da je verižni ulomek $[a_0, a_1,...]$ enak $e$.\par Vse tri rekurzije želimo zmanjšati na eno samo rekurzijo, ki bo vključevala le člene $p_{3n}$, $p_{3n-3}$ in $p_{3n-6}$. Z nekaj substitucij dobimo rekurzijo
\[ p_{3n} = 2(2n-1)p_{3n-3}+p_{3n-6}, \]
\[ q_{3n} = 2(2n-1)q_{3n-3}+q_{3n-6}. \]

Vemo, da obstaja limita delnih konvergentov tega verižnega ulomka, saj je enostaven. Dovolj je pokazati, da k $e$ konvergira podzaporedje $\{c_{3n}\}$, saj s tem dobimo stekališče konvergentnega zaporedja, ki je hkrati tudi limita tega zaporedja. Uvedemo torej $x_n = p_{3n}$ in $y_n = q_{3n}$. Velja:
\[ x_n = 2(2n-1)p_{n-1}+p_{n-2}, \]
\[ y_n = 2(2n-1)q_{n-1}+q_{n-2}. \]

Iz zapisa prvih nekaj členov verižnega ulomka za $e$ pa tudi izračunamo:
\[ x_0 = 1,\quad y_0=1,\quad x_1=3,\quad y_1=1.\]

Sedaj vpeljemo zaporedje integralov $$T_n = \int_{0}^{1}\frac{t^n(t-1)^n}{n!}e^t dt.$$
Izračunamo:
$$T_0 = \int_{0}^{1}e^t dt = e -1$$
$$T_1 = \int_{0}^{1}t(t-1)e^t dt=e-3$$

\begin{lem}
\label{rekurzija}
Naj bo $T_n$, $x_n$ in $y_n$ kot zgoraj. Tedaj velja $T_n=y_n e-x_n$.
\end{lem}
\begin{proof}
Za $n=0,1$ vidimo resničnost leme zgoraj. Predpostavimo, da lema velja za $T_{n-1}$. Z integracijo po delih integrala $T_n$ dobimo 
\begin{equation*}
\begin{split}
T_n &= 2(2n-1)T_{n-1}+T_{n-2} \\
&=2(2n-1)(e y_{n-1} -x_{n-1}) + e y_{n-2} -x_{n-2} \\
&=(2(2n-1)y_{n-1}+y_{n-2})e - (2(2n-1)x_{n-1}+x_{n-2}) \\
&=y_n e - x_n.
\end{split}
\end{equation*}
\end{proof}

Očitno velja $\lim_{n\to\infty}T_n =0$, saj funkcije pod integralom enakomerno konvergirajo k 0, torej je po Lemi \ref{rekurzija}:
\[ \lim_{n\to\infty}(y_ne-x_n) = 0 \Rightarrow \lim_{n\to\infty}\frac{x_n}{y_n}=\lim_{n\to\infty}(e-\frac{T_n}{y_n}) =e. \]

Naš verižni ulomek je torej res število $e$. V nasprotju s tem število
$$\pi=[3, 7, 15, 1, 292, 1, 1, 1, 2, 1, 3, 1, 14, 2, 1, 1, 2, 2, 2, 2, ...]$$
nima znanega posebno lepega zaporedja $\{a_n\}$, čeprav je izračunanih mnogo členov. Zanimivo pa je, da imata denimo $\frac{4}{\pi}$ in $\frac{\pi}{2}$ zelo očitne vzorce v posplošenih verižnih ulomkih tj. v verižnih ulomkih, katerih števci niso nujno enaki 1.

\section{Kvadratne iracionale in periodični verižni ulomki} \label{periode}

Najprej definirajmo pojma, ki jih bomo spoznavali in povezali v tem poglavju.

\begin{df}
Realno število $\alpha$ je \textit{kvadratna iracionala}, če je iracionalno in rešitev neke kvadratne enačbe s koeficienti iz $\Q$.
\end{df}
\begin{df}
Verižni ulomek $[a_0, a_1, a_2, ...]$ je \textit{periodičen}, če obstaja tak $h>0$, da je $$a_n=a_{n+h}$$ za vse dovolj velike $n$.
\end{df}

Primer periodičnega verižnega ulomka je denimo $x = [\overline{2}] =[2,2,2,2,...]$. Zanj velja
\begin{align*}
x&= 2 + \frac{1}{x} \\
x&= \frac{2x+1}{x} \\
x^2&= 2x+1 \\
0&=x^2-2x-1\\
x&=\frac{2+\sqrt{4+4}}{2}\\
x&=1+\sqrt{2}.
\end{align*}

Vidimo, da je ta verižni ulomek kvadratna iracionala, kar nas napelje na idejo, da je vsak periodičen verižni ulomek možno zapisati s kvadratno iracionalo.

\begin{iz}
\label{periodicni}
\textit{Karakterizacija periodičnega verižnega ulomka}: Realno število $\alpha$ je kvadratna iracionala natanko tedaj, ko je njegov zapis z verižnim ulomkom periodičen.
\end{iz}
\begin{proof}
($\Leftarrow$) Brez škode za splošnost lahko predpostavimo $\alpha = [\overline{a_0, a_1, ..., a_n}]$. Tedaj je $\alpha = [a_0, a_1, ..., a_n, \alpha]$. Po (\ref{pninqn}) velja \[ \alpha = \frac{\alpha p_{n-1} + p_{n-2}}{\alpha q_{n-1} + q_{n-2}}. \]
S poenostavitvijo te enačbe pokažemo, da $\alpha$ zadostuje kvadratni enačbi z racionalnimi koeficienti. Ker je ulomek neskončen, $\alpha$ ne more biti racionalno število. Zato je $\alpha$ kvadratna iracionala.\\
($\Rightarrow$) gl. \cite{teorijastevil}, str. 113
\end{proof}

Smiselno se nam porodi vprašanje, kakšni so verižni ulomki algebraičnih števil višje stopnje, denimo $$\sqrt[3]{2} =[1, 3, 1, 5, 1, 1, 4, 1, 1, 8, 1, 14, 1, 10, 2, 1, 4, 12, 2, 3, 2, 1, 3, 4, 1, 1, 2, 14,
3, ...],$$
kjer ni vidnega nobenega očitnega zaporedja. Povezava med algebraičnimi števili višje stopnje in verižnimi ulomki je odprt problem in predmet različnih razprav.
\newpage

\section{Zaključek}

V svoji seminarski nalogi sem v grobem zajel vse osnovne izreke in trditve teorije verižnih ulomkov, ki so se izkazali kot uspešen način za predstavitev realnih števil. Težava pri uporabi verižnih ulomkov so seveda računske operacije, saj je seštevanje zelo dolgih verižnih ulomkov dolgotrajno, prav tako množenje.\par
Teorija in primeri verižnih ulomkov segajo daleč nazaj v čas Evklida - videli smo tudi, kako lahko njegov algoritem s pridom uporabimo - nato pa so ulomke preučevali mnogi veliki matematiki skozi vso zgodovino. Euler je na primer dokazal iracionalnost števila $e$ tako, da je pokazal, da je njegov verižni ulomek neskončen. V zadnjem poglavju smo se tudi srečali z odprtim problemom, ki nakazuje, da so verižni ulomki še vedno področje raziskovanja.\par

\begin{thebibliography}{5}
\bibitem{teorijastevil}\textit{William Stein: Elementary Number Theory:
Primes, Congruences, and Secrets,} UC San Diego, Harvard, University of Washington, 2017
\bibitem{euler}\textit{Henry Cohn: A Short Proof of the Simple Continued Fraction Expansion of e,} The American Mathematical Monthly, Vol. 113, No. 1 (Jan., 2006), pp. 57-62
\bibitem{uporaba}\textit{Michel Waldschmidt: Continued fractions:
introduction and applications,}, The Ninth International Conference on
Science and Mathematics Education in Developing Countries (predstavitev), Mandalay University, 2016
\end{thebibliography}

\end{document}